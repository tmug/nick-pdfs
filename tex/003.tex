\documentclass{article}
\usepackage[a4paper]{geometry}
\usepackage{parskip}
\usepackage[colorlinks=true,linkcolor=black,urlcolor=blue]{hyperref}
\usepackage{amsmath}
\usepackage{tikz}
\usepackage{listings}
\urlstyle{same}
\date{}
\author{Sunaina Pai}

\title{Two Logicians}
\begin{document}
\maketitle


\section*{Problem}
Two perfect logicians, S and P, are told that integers \( x \) and \( y \)
have been chosen such that \( 1 < x < y \) and \( x + y < 100 \). S is
given the value \( x + y \) and P is given the value \( xy \). They then
have the following conversation.

\setlength{\leftskip}{2em}
P:  I cannot determine the two numbers. \\
S:  I knew that. \\
P:  Now I can determine them. \\
S:  So can I.
\setlength{\leftskip}{0em}

Given that the above statements are true, what are the two numbers?
(Computer assistance allowed.)

Source: \url{http://www.qbyte.org/puzzles/puzzle01.html#p3}


\section*{Solution}
This problem is solved using a computer program written in Python. The
program and its output, respectively, are available at the following
URLs:
\begin{itemize}
\item \url{https://github.com/sunainapai/lab/blob/master/math/nick/003.py}
\item \url{https://github.com/sunainapai/lab/blob/master/math/nick/003.txt}
\end{itemize}

The program and the output are also included in the next two sections.

The output shows that the solution is
\begin{align*}
x & = 4 \\
y & = 13
\end{align*}
\pagebreak


\section*{Program Code}
\lstset{language=Python,basicstyle=\small\ttfamily,columns=fullflexible,keepspaces,breaklines}
\lstinputlisting{003.py}


\section*{Program Output}
\lstset{language=}
\lstinputlisting{003.txt}


\end{document}
